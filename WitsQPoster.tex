%\title{Wits Q poster}
\documentclass[a0,portrait]{a0poster}
\usepackage{a0size}
\usepackage{multicol} % This is so we can have multiple columns
    \columnsep=100pt % White space between the columns
    \columnseprule=3pt % Thickness of the black line between the columns
    
% \usepackage{pgfpages}
% \pgfpagesdeclarelayout{resize and center}
% {
%   \def\pgfpageoptionborder{0pt}
% }
% {
%   \pgfpagesphysicalpageoptions
%   {%
%     logical pages=1,%
%     physical height=\pgfpageoptionheight,%
%     physical width=\pgfpageoptionwidth%
%   }
%   \pgfpageslogicalpageoptions{1}
%   {%
%     resized width=\pgfphysicalwidth,%
%     resized height=\pgfphysicalheight,%
%     border shrink=\pgfpageoptionborder,%
%     center=\pgfpoint{.5215\pgfphysicalwidth}{.47\pgfphysicalheight}%
%   }%
% }
% \pgfpagesuselayout{resize and center}[a2paper,landscape]


\usepackage[svgnames]{xcolor} % Specify colors by their 'svgnames'
\usepackage{times} % Use the times font
\usepackage{graphicx}
    \graphicspath{{figures/}} % Location of the graphics files
\usepackage{booktabs} % Top and bottom rules for table
\usepackage[font=small,labelfont=bf]{caption}
\usepackage{amsfonts, amsmath, amsthm, amssymb}
\usepackage{wrapfig} % Allows wrapping text around tables and figures

\begin{document}
%----------------------------------------------------------------------------------------
%	POSTER HEADER 
%----------------------------------------------------------------------------------------
%Left Logo
\begin{minipage}[c]{0.08\linewidth}%
\includegraphics[width=\linewidth]{witslogo}\\%
\end{minipage}%
%Central
\begin{minipage}[c]{0.78\linewidth}%
\centering%
%Main Title
\veryHuge \color{NavyBlue} \textbf{How can Quantum Computing improve practical  Machine Learning today?}%
%Sub title
 \color{Black}\\[0.3cm]%
\Huge\textit{An exploration of how near term quantum computing can practically be used for machine learning applications}\\[1.2cm]%
%Authors
\large \textbf{Tau Merand\textsuperscript{$\dagger$} \& David Merand\textsuperscript{$\ddagger$}}\\[0.5cm]%
%Institutes
\normalsize \textsuperscript{$\dagger$}School of Computer Science and Applied Mathematics, University of the Witwatersrand\\%
\textsuperscript{$\dagger$}Academic Development Unit, Faculty of Engineering and Built Environment, University of the Witwatersrand\\%
%Contacts
\texttt{tau.taylor.merand@gmail.com david.merand@wits.ac.za}\\%
\end{minipage}%
%Right Logo
\begin{minipage}[t]{0.12\linewidth}%
\includegraphics[width=\linewidth]{Wits-Q-logo.png}%
\end{minipage}%

\vspace{1.5cm}% Whitespace between the header and poster content

%----------------------------------------------------------------------------------------
% Body of Poster
\begin{multicols}{2} % This is how many columns your poster will be broken into, a portrait poster is generally split into 2 columns
%----------------------------------------------------------------------------------------
\color{Navy} % Navy color for the abstract
%\begin{abstract}
\section*{Abstract}
Quantum Computing (QC) has grown in leaps and bounds in the last 5-10 years; both in hardware and software .  Machine Learning (ML) is a field where large data-sets and massively parallel computing architectures have allowed for similar rapid improvements. We propose the use of current and near term QC as viable technology for boosting existing classical ML algorithms. The exponential nature of super-positioned qubits can be leveraged to perform classically hard optimisations efficiently, however many existing difficulties in QC hardware are not clearly tractable.  Therefore while the development of Quantum ML (QML) techniques has been an incredible area of active research, quantum computing hardware has not yet reached the capability to run anything more realistic than the simplest proof-of-concept type examples, due to limitations in memory, qubit capacity and coherence time.  We suggest that more actionable improvements may be made in ML by focusing on leveraging QC on smaller, tangential optimisation problems to the actual learning rather than solely trying to learn or implement learning algorithms on a QC. We suspect that using QC to either select effective model meta-learning parameters or to select useful initial values combined with traditional training of the resultant model on large data sets may yield better or faster learning while being feasible to implement now or in the very near future. Hopefully this will be a more immediately actionable approach to improving existing ML on practical problems. Good results have been achieved by QC on combinatorial optimisation problems so it seems promising to reframe meta-learning parameter or initial value selection as a in this way and then execute a classical training-on-data gradient descent type algorithm on classical computing architectures. This approach seems likely to yield practical speed-ups or improvements in the ML sphere while QC hardware is still ill equipped to deal with real data sets.
%\end{abstract}
%----------------------------------------------------------------------------------------
\color{SaddleBrown} % SaddleBrown color for the introduction
\section*{Introduction}


%----------------------------------------------------------------------------------------
\color{DarkSlateGray} % DarkSlateGray color for the rest of the content

\section*{The Present and Future of Quantum Computing Hardware}

\begin{minipage}[c]{0.45\linewidth}%
	\includegraphics[width=\linewidth]{MooresLaw}\\%
\end{minipage}%
\begin{minipage}[c]{0.45\linewidth}%
	\includegraphics[width=\linewidth]{Qvolume}\\%
\end{minipage}%
\captionof{figure}{\color{Green}Moores's Law for classical architectures vs IBM's goal for Quantum Volume}
%----------------------------------------------------------------------------------------

\section*{The State of Quantum Computing Optimisation  Algorithms}%The?


%----------------------------------------------------------------------------------------
\section*{The Problems and Drawbacks of Quantum ML}
Nulla vel nisl sed mauris auctor mollis non sed. 

\begin{equation}
E = mc^{2}
\label{eqn:Einstein}
\end{equation}

Curabitur mi sem, pulvinar quis aliquam rutrum. (1) edf (2)
, $\Omega=[-1,1]^3$, maecenas leo est, ornare at. $z=-1$ edf $z=1$ sed interdum felis dapibus sem. $x$ set $y$ ytruem. 
Turpis $j$ amet accumsan enim $y$-lacina; 
ref $k$-viverra nec porttitor $x$-lacina. 

Vestibulum ac diam a odio tempus congue. Vivamus id enim nisi:

\begin{eqnarray}
\cos\bar{\phi}_k Q_{j,k+1,t} + Q_{j,k+1,x}+\frac{\sin^2\bar{\phi}_k}{T\cos\bar{\phi}_k} Q_{j,k+1} &=&\nonumber\\ 
-\cos\phi_k Q_{j,k,t} + Q_{j,k,x}-\frac{\sin^2\phi_k}{T\cos\phi_k} Q_{j,k}\label{edgek}
\end{eqnarray}
and
\begin{eqnarray}
\cos\bar{\phi}_j Q_{j+1,k,t} + Q_{j+1,k,y}+\frac{\sin^2\bar{\phi}_j}{T\cos\bar{\phi}_j} Q_{j+1,k}&=&\nonumber \\
-\cos\phi_j Q_{j,k,t} + Q_{j,k,y}-\frac{\sin^2\phi_j}{T\cos\phi_j} Q_{j,k}.\label{edgej}
\end{eqnarray} 

Nulla sed arcu arcu. Duis et ante gravida orci venenatis tincidunt. Fusce vitae lacinia metus. Pellentesque habitant morbi. $\mathbf{A}\underline{\xi}=\underline{\beta}$ Vim $\underline{\xi}$ enum nidi $3(P+2)^{2}$ lacina. Id feugain $\mathbf{A}$ nun quis; magno.

%----------------------------------------------------------------------------------------
\section*{Quantum Computing for Initialisation}%
Phasellus imperdiet, tortor vitae congue bibendum, felis enim sagittis lorem, et volutpat ante orci sagittis mi. Morbi rutrum laoreet semper. Morbi accumsan enim nec tortor consectetur non commodo nisi sollicitudin. Proin sollicitudin. Pellentesque eget orci eros. Fusce ultricies, tellus et pellentesque fringilla, ante massa luctus libero, quis tristique purus urna nec nibh.

Nulla ut porttitor enim. Suspendisse venenatis dui eget eros gravida tempor. Mauris feugiat elit et augue placerat ultrices. Morbi accumsan enim nec tortor consectetur non commodo. Pellentesque condimentum dui. Etiam sagittis purus non tellus tempor volutpat. Donec et dui non massa tristique adipiscing. Quisque vestibulum eros eu. Phasellus imperdiet, tortor vitae congue bibendum, felis enim sagittis lorem, et volutpat ante orci sagittis mi. Morbi rutrum laoreet semper. Morbi accumsan enim nec tortor consectetur non commodo nisi sollicitudin.



In hac habitasse platea dictumst. Etiam placerat, risus ac.

Adipiscing lectus in magna blandit:

\begin{center}\vspace{1cm}
\begin{tabular}{l l l l}
\toprule
\textbf{Treatments} & \textbf{Response 1} & \textbf{Response 2} \\
\midrule
Treatment 1 & 0.0003262 & 0.562 \\
Treatment 2 & 0.0015681 & 0.910 \\
Treatment 3 & 0.0009271 & 0.296 \\
\bottomrule
\end{tabular}
\captionof{table}{\color{Green} Table caption}
\end{center}\vspace{1cm}

Vivamus sed nibh ac metus tristique tristique a vitae ante. Sed lobortis mi ut arcu fringilla et adipiscing ligula rutrum. Aenean turpis velit, placerat eget tincidunt nec, ornare in nisl. In placerat.

\begin{center}\vspace{1cm}
\includegraphics[width=0.8\linewidth]{placeholder}
\captionof{figure}{\color{Green} Figure caption}
\end{center}\vspace{1cm}

%----------------------------------------------------------------------------------------
\section*{Quantum computing for Parameter Selection}
Aliquam non lacus dolor, \textit{a aliquam quam} \cite{Smith:2012qr}. Cum sociis natoque penatibus et magnis dis parturient montes, nascetur ridiculus mus. Nulla in nibh mauris. Donec vel ligula nisi, a lacinia arcu. Sed mi dui, malesuada vel consectetur et, egestas porta nisi. Sed eleifend pharetra dolor, et dapibus est vulputate eu. \textbf{Integer faucibus elementum felis vitae fringilla.} In hac habitasse platea dictumst. Duis tristique rutrum nisl, nec vulputate elit porta ut. Donec sodales sollicitudin turpis sed convallis. Etiam mauris ligula, blandit adipiscing condimentum eu, dapibus pellentesque risus.

\textit{Aliquam auctor}, metus id ultrices porta, risus enim cursus sapien, quis iaculis sapien tortor sed odio. Mauris ante orci, euismod vitae tincidunt eu, porta ut neque. Aenean sapien est, viverra vel lacinia nec, venenatis eu nulla. Maecenas ut nunc nibh, et tempus libero. Aenean vitae risus ante. Pellentesque condimentum dui. Etiam sagittis purus non tellus tempor volutpat. Donec et dui non massa tristique adipiscing.
%----------------------------------------------------------------------------------------
\color{SaddleBrown} % SaddleBrown color for the conclusions to make them stand out
\section*{Conclusions}
\begin{itemize}
\item Pellentesque eget orci eros. Fusce ultricies, tellus et pellentesque fringilla, ante massa luctus libero, quis tristique purus urna nec nibh. Phasellus fermentum rutrum elementum. Nam quis justo lectus.
\item Vestibulum sem ante, hendrerit a gravida ac, blandit quis magna.
\item Donec sem metus, facilisis at condimentum eget, vehicula ut massa. Morbi consequat, diam sed convallis tincidunt, arcu nunc.
\item Nunc at convallis urna. isus ante. Pellentesque condimentum dui. Etiam sagittis purus non tellus tempor volutpat. Donec et dui non massa tristique adipiscing.
\end{itemize}

\color{DarkSlateGray} % Set the color back to DarkSlateGray for the rest of the content
%----------------------------------------------------------------------------------------

\section*{Further Research}

Vivamus molestie, risus tempor vehicula mattis, libero arcu volutpat purus, sed blandit sem nibh eget turpis. Maecenas rutrum dui blandit lorem vulputate gravida. Praesent venenatis mi vel lorem tempor at varius diam sagittis. Nam eu leo id turpis interdum luctus a sed augue. Nam tellus.

 %----------------------------------------------------------------------------------------

\nocite{*} % Print all references regardless of whether they were cited in the poster or not
\bibliographystyle{plain} % Plain referencing style
\bibliography{biblio} % Use the example bibliography file sample.bib

%----------------------------------------------------------------------------------------
%	ACKNOWLEDGEMENTS
%----------------------------------------------------------------------------------------

\section*{Acknowledgements}

Etiam fermentum, arcu ut gravida fringilla, dolor arcu laoreet justo, ut imperdiet urna arcu a arcu. Donec nec ante a dui tempus consectetur. Cras nisi turpis, dapibus sit amet mattis sed, laoreet.

%----------------------------------------------------------------------------------------
\end{multicols}


\end{document}